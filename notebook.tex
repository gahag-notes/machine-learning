% Created 2019-03-26 Tue 18:19
% Intended LaTeX compiler: pdflatex
\documentclass[11pt]{article}
\usepackage[utf8]{inputenc}
\usepackage[T1]{fontenc}
\usepackage{graphicx}
\usepackage{grffile}
\usepackage{longtable}
\usepackage{wrapfig}
\usepackage{rotating}
\usepackage[normalem]{ulem}
\usepackage{amsmath}
\usepackage{textcomp}
\usepackage{amssymb}
\usepackage{capt-of}
\usepackage{hyperref}
\usepackage{minted}
\usepackage[margin=2cm]{geometry}
\DeclareMathOperator{\sign}{sign}
\setlength{\parindent}{0cm}
\date{\today}
\title{}
\hypersetup{
 pdfauthor={},
 pdftitle={},
 pdfkeywords={},
 pdfsubject={},
 pdfcreator={Emacs 26.1 (Org mode 9.2.2)}, 
 pdflang={English}}
\begin{document}


\section{Supervised learning}
\label{sec:org0ec48b1}
Method where you train the program by feeding the learning algorithm with a mapping of
inputs to correct outputs.
\subsection{Regression}
\label{sec:orgab8086c}
Regression is curve fitting: learn a continuous input \(\to\) output mapping from a set of
examples.
\subsection{Classification}
\label{sec:org81aa301}
Outputs are discrete variables (category labels). Learn a decision boundary that
separates one class from the other. Generally, a confidence is also desired, i.e.,
how sure are we that the input belongs to the chosen category.
\subsection{Training set}
\label{sec:org0fe21d5}
The training set is a set of \(m\) \((X,\, y)\) pairs, where:
\begin{align*}
  X \in \mathbb{R}^d & \quad\text{models the input.} \\
  y \in \{0, 1\} & \quad\text{models the output.}
\end{align*}
\subsection{Error function}
\label{sec:orgb1c2268}
The error function for a model \(f: X \mapsto y\) parameterized by \(W\) applied to a
dataset \(\{ (X,\, y) \}\) of size \(m\) is:
\[
  \min_W \enspace \sum^m_{i=1}{ \big(f_W(X_i) - y_i \big)^2 }
\]
\subsection{Perceptron}
\label{sec:org73f3032}
Perceptron is the trivial neural network. The model for a parameter \(W = (\text{threshold},\,
   w_1,\, \hdots,\, w_d)\) and inputs of the form \((1,\, x_1,\, \hdots,\, x_d)\) is given by
\[
  f_W(X) = \sign(W^{\top} X)
\]
If \(x_i\) is evidence for approval, then \(w_i\) should be high. \\
If \(x_i\) is evidence for denial, then \(w_i\) should be low.
\subsubsection{Learning algorithm}
\label{sec:orgc0e8beb}
The learning algorithm of the Perceptron is quite simple. The learning rate \(\in (0,\,
    1]\) is used to scale each step. the For a training set \(S = \{ \, (X_1,\, y_1),\enspace (X_1,\,
    y_1),\enspace \hdots \, \}\)
\begin{itemize}
\item Starting with random weights, show each sample in sequence repetitively.
\item If the output is correct, do nothing.
\item If the produced output is negative, and the correct output is positive, increase the weights.
\item If the produced output is positive, and the correct output is negative, decrease the weights.
\item The amount to increase/decrease is given by the current sample scaled by the learning rate.
\end{itemize}
\subsection{Error}
\label{sec:org078ee5d}
The error function for a model \(f\) in a \textbf{training} sample is
\[ E_{\text{in}}(f) \]
This function is known and calculable.
\\[10pt]
The error function for a model \(f\) in a \textbf{test} sample is
\[ E_{\text{ou}t}(f) \]
This function is \textbf{not} known, and only \textbf{approachable}.
\\[10pt]
Given a model \(f\) in a set of \(M\) models, the bound for the probability of the error
deviation surpassing a given \(\epsilon\) is
\[
  \mathbb{P}\left(\big| E_{\text{in}}(f) - E_{\text{ou}t}(f) \big| > \big\epsilon\right) \leq 2Me^{-2N\big\epsilon^2}
\]
Notably, \(E_{\text{in}}(f)\) and \(E_{\text{out}}(f)\) deviates as \(f\) becomes complex.
\subsubsection{Empirical error minimization}
\label{sec:org26a96e5}
During the learning algorithm, always conserve the weights that produce the lower error. \\
This has a disadvantage: It memorizes the training set.
\subsection{Ensemble learning}
\label{sec:org7efa6c1}
Ensemble learning consists in combining several simple models to form a more complex
model.
\begin{description}
\item[{Bagging:}] Each model training with a different dataset
\item[{Boosting:}] Same dataset, but instrumented for each model to mitigate the weakness of
others
\end{description}
\subsection{Learning decision trees}
\label{sec:org031237a}
Each layer is consists of an attribute that splits the data into subsets that are
ideally disjoint.
\section{Reinforcement learning}
\label{sec:orgc988d77}
Method where you train the program by rewarding the learning algorithm positively or
negatively according to the produced results. This method is similar to how we teach
animals.
\section{Unsupervised learning}
\label{sec:orgf13811b}
Given only inputs as training, find a pattern: discover clusters, manifolds, embedding.
\end{document}
